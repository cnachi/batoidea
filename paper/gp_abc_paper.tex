% This document is part of the ShiftyLines project.
% Copyright 2016 the authors.

\documentclass[12pt]{emulateapj}
\usepackage{graphicx}
%\usepackage{epsfig}
\usepackage{times}
\usepackage{natbib}
\usepackage{amsfonts}
\usepackage{amsmath}
\usepackage{amsbsy}
\usepackage{bm}
\usepackage{hyperref}
\usepackage{url}
%\usepackage{subfigure}
\usepackage{microtype}
\usepackage{rotating}
\usepackage{booktabs}
\usepackage{threeparttable}
\usepackage{tabularx}
\usepackage{subfigure}
\DeclareMathOperator\erf{erf}


%\usepackage{longtable}%\usepackage[stable]{footmisc}
%\usepackage{color}
%\bibliographystyle{apj}

\newcommand{\project}[1]{\textsl{#1}}
\newcommand{\fermi}{\project{Fermi}}
\newcommand{\rxte}{\project{RXTE}}
\newcommand{\chandra}{\project{Chandra}}
\newcommand{\athena}{\project{Athena+}}
\newcommand{\xmm}{\project{XMM-Newton}}
\newcommand{\given}{\,|\,}
\newcommand{\dd}{\mathrm{d}}
\newcommand{\counts}{y}
\newcommand{\pars}{\theta}
\newcommand{\mean}{m}
\newcommand{\likelihood}{{\mathcal L}}
\newcommand{\Poisson}{{\mathcal P}}
\newcommand{\Uniform}{{\mathcal U}}
\newcommand{\bkg}{\mathrm{bkg}}
\newcommand{\word}{\phi}

%\usepackage{breqn}

%\newcommand{\bs}{\boldsymbol}

\begin{document}

\title{Precise Probabilistic Inference of Spectral-Timing Properties from Multi-Wavelength Observations of Black Holes}

%\author{}
 
%  \altaffiltext{1}{Center for Data Science, New York University, 726 Broadway, 7th Floor, New York, NY 10003}
%  \altaffiltext{2}{{\tt daniela.huppenkothen@nyu.edu}}

\begin{abstract}
Understanding the formation and growth of black holes is of key importance to our knowledge of both stellar evolution as 
well as of how galaxies form and evolve. In recent years spectral-timing, the combined use of both temporal and wavelength information,
has become the standard in inferring physical properties like mass and spin of these systems. Current state-of-the-art techniques 
heavily rely on Fourier analysis to derive important observables like time lags, power spectra and coherence estimates. However, these 
methods will rapidly become unreliable in the face of future multi-wavelength surveys that largely rely on an irregular sampling of observations.
Alternative methods can currently only model light curves at different wavelengths independently, or make strong assumptions about the 
underlying emission processes. Here, we explore the use of fast Gaussian Process inference combined with Approximate Bayesian 
Computation to build probabilistic models of multi-wavelength variability of black hole light curves. We show that (1) Gaussian Processes 
can effectively model the light curve at different wavelengths and derive accurate time lags even in unevenly sampled data, and (2) in 
the absence of full physical models for these systems, Approximate Bayesian Computation allows us to measure important properties of the 
data, including power spectral shape, time lags, coherence, and flux distribution simultaneously.

\end{abstract}

\keywords{methods:statistics}

\section{Introduction}

\bibliography{td}
\bibliographystyle{apj}

\end{document}


